\documentclass[11pt]{article}

\usepackage{../analysis}

\begin{document}

\coverpage{6}

% hw problem 1 -----------------------------------------------------------------

\begin{exercise}{3.3.1}{1}
    \problem{
        Show that compact sets are closed under arbitrary intersections and finite unions.
    }
    \proof{
        First note that a set of real numbers $A$ is compact if and only if $A$ is closed and bounded.
        Additionally, note that closed sets are closed under arbitrary intersection and finite unions. \parspace
        We now show that compact sets are closed under arbitrary intersections.
        Let $\B$ be any collection of compact sets and $B' = \cup _{B \in \B} B$.
        The set $B'$ is compact if and only if it is closed and bounded.
        Certainly $\B$ is closed since every $B \in \B$ is closed and their intersection must be closed.
        We show that $B'$ is bounded by contradiction.
        Suppose $B'$ is not bounded.
        Then there exists a sequence $x_1, x_2, ...$ of elements in $B'$ such that for all $n$, there exists a $j$ such that $x_j > n$ or $x_j < -n$.
        But sequence $x_1, x_2, ...$ must also be in every $B \in \B$ so we have just shown that every $B \in \B$ is unbounded.
        But then no $B \in \B$ is compact, which is a contradiction.
        Therefore $B'$ is both closed and bounded, which implies that $B'$ is compact. \parspace
        We now show that compact sets are closed under finite unions.
        Let $C_1, C_2, ..., C_n$ be a finite collection of compact sets and $C' = \cap _{i = 1}^n C_i$.
        To show that $C'$ is compact, we will show that it is closed and bounded.
        $C'$ is closed since it is the intersection of a finite collection of closed sets.
        Is $C'$ bounded?
        Suppose that $C'$ is not bounded, then one of two cases must occur.
        Either there exists a monotonically increasing sequence $x_1, x_2, ...$ of elements in $C'$ such that for all $n$ there exists $j$ such that $x_j > n$ or there exists a monotonically decreasing sequence $y_1, y_2, ...$ of elements in $C'$ such that for all $n$ there exists $j$ such that $y_j < -n$.
        In either case, we can use the pigeon hole principle to say at least one of $B_i$ contains an infinite number of terms in the sequence (whether it be increasing or decreasing).
        But then that $B_i$ must be unbounded, which is a contradiction since we assumed every $B_i$ to be compact.
        Therefore, the union of a finite number of compact sets must be compact.
    }
\end{exercise}

% hw problem 2 -----------------------------------------------------------------

\begin{exercise}{3.3.1}{4}
    \problem{
        If $A \subset B_1 \cup B_2$ where $B_1$ and $B_2$ are disjoint open sets and $A$ is compact, show that $A \cap B_1$ is compact.
        Is the same true if $B_1$ and $B_2$ are not disjoint?
    }
    \proof{
        Let's first analyze the case where $B_1 \cap B_2 = \emptyset$.
        So we must show that $A \cap B_1$ is compact, that is, every sequence that lies entirely in $A \cap B_1$ has a limit-point in $A \cap B_1$.
        To this end, pick a sequence $x_1, x_2, ...$ that lies entirely in $A \cap B_1$.
        Every $x_j$ is a point in $A$ and $A$ is compact so $x_1, x_2, ...$ certainly has a limit-point $x$ in $A$.
        But is $x$ a member of $B_1$?
        Since $A \subset B_1 \cup B_2$ and $x \in A$, it must be true that $x$ is in $B_1$ or $B_2$.
        Further, we know that $B_1 \cap B_2 = \emptyset$ so $x$ must be in only one of $B_1$ and $B_2$. \parspace
        Suppose that $x$ is a point in $B_2$.
        Since $B_2$ is open, there must exist an open interval $(a,b) \subset B_2$ such that $x \in (a,b)$.
        Since $(a,b) \subset B_2$ and $B_1 \cap B_2 = \emptyset$, no elements of $B_1$ are in the interval $(a,b)$.
        But every element in the sequence $x_1, x_2, ...$ is a member of $B_1$, so we have just shown that there exists a neighborhood $(a,b)$ of $x$ such that no element of $x_1, x_2, ...$ is contained $(a,b)$.
        Then $x$ is not a limit-point of $x_1, x_2, ...$, but this is a contradiction since we know $x$ to be a limit-point of the sequence.
        So $x \notin B_2$, which implies that $x \in B_1$.
        Then $x$ (a limit-point of $x_1, x_2, ...$) is a point in both $A$ and $B_1$, which implies that $x \in A \cap B_1$.
        Since $x_1, x_2, ...$ is an arbitrary sequence entirely in $A \cap B_1$, we have just shown that every sequence in $A \cap B_1$ has a limit-point in $A \cap B_1$.
        In other words, we have shown that $A \cap B_1$ is compact. \parspace
        We now wonder if $A \cap B_1$ is necessarily compact if we instead assume that $B_1 \cap B_2 \neq \emptyset$.
        The statement does not hold; for evidence, see the following counterexample.
        Let $B_1 = (0,2)$ and $B_2 = (1, 3)$ (open sets with a nonempty intersection).
        Then let $A = [1,2] \subset B_1 \cup B_2$ (which is compact since $A$ is closed and bounded).
        Then $A \cap B_1 = [1,2)$.
        The set $A \cap B_1$ is not closed since it does not contain its limit-ponit 2, then since $A \cap B_1$ is not closed, it is not compact.
    }
\end{exercise}

% hw problem 3 -----------------------------------------------------------------

\begin{exercise}{3.3.1}{8}
    \problem{
        If $A$ is compact, show that $\sup A$ and $\inf A$ belong to $A$.
        Give an example of a non-compact set $A$ such that both $\sup A$ and $\inf A$ belong to $A$.
    }
    \proof{
        Suppose that we have some compact set $A$, we must show that $\sup A \in A$ and $\inf A \in A$.
        First we will show that $\sup A \in A$.
        Let $y = \sup A$.
        Then for all $n$, there exists some $y_n \in A$ such that $y - 1/n < y_n$.
        Which implies that for all $n$, there exists $y_n \in A$ such that $y - y_n = |y - y_n| < 1/n$.
        Let $y_1, y_2, ...$ be a sequence (not necessarily distinct) that satisfies $|y - y_n| < 1/n$.
        From this, we can gather that $\{ y_n \}$ converges to $y$ (which means that $y$ is the only limit point of the sequence).
        Then since every $y_n \in A$ and $A$ is compact, some limit-point of $y_1, y_2, ...$ is a point in $A$.
        But we already know that $y$ is the only limit-point of $y_1, y_2, ...$, so we must have $y \in A$.
        And $y = \sup A$ so it must be true that $\sup A \in A$. \parspace
        Similarly, we now show that $\inf A \in A$.
        Let $z = \inf A$.
        Then for all $n$, there exists some $z_n \in A$ such that $z + 1/n > z_n$.
        Which implies that for all $n$, there exists $z_n \in A$ such that $ z_n - z = |z_n - z| = |z - z_n | < 1/n$.
        Let $z_1, z_2, ...$ be a sequence (not necessarily distinct) that satisfies $|z - z_n| < 1/n$.
        From this, we can gather that $\{ z_n \}$ converges to $z$ (which means that $z$ is the only limit point of the sequence).
        Then since every $z_n \in A$ and $A$ is compact, some limit-point of $z_1, z_2, ...$ is a point in $A$.
        But we already know that $z$ is the only limit-point of $z_1, z_2, ...$, so we must have $z \in A$.
        And $z = \inf A$ so it must be true that $\inf A \in A$. \parspace
        We now give an example of a non-compact set $A$ such that $\sup A$ and $\inf A$ belong to $A$.
        Let $A = [0,1) \cup (1, 2]$.
        The set $A$ is not closed since it does not contain its limit-point 1, which means $A$ is not compact.
        However, $\sup A = 2 \in A$ and $\inf A = 0 \in A$.
    }
\end{exercise}

\end{document}
