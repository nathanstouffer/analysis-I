\documentclass[11pt]{article}

\usepackage{../analysis}

\begin{document}

\coverpage{3}

% hw problem 1 -----------------------------------------------------------------

\begin{exercise}{2.2.4}{3}
    \problem{
        If $x$ is a real number, show that there exists a Cauchy sequence of rationals $x_1, x_2, ...$ representing $x$ such that $x_n < x$ for all $n$.
    }
    \proof{

    }
\end{exercise}

% hw problem 2 -----------------------------------------------------------------

\begin{exercise}{2.2.4}{7}
    \problem{
        Prove that $|x - y| \geq |x| - |y|$ for any real numbers $x$ and $y$.
    }
    \proof{
        We must show that $|x - y| \geq |x| - |y|$ holds for all real numbers $x$ and $y$.
        We take $|x+y| \leq |x| + |y|$ (the triangle inequality) to be true.
        Then let $x$ and $y$ be any real numbers.
        Certainly it is true that $|x| = |x - y + y|$ since $\R$ is an ordered field.
        Also, by the triangle inequality, $|x - y + y| \leq |x-y| + |y|$.
        Since $|x - y + y| = |x|$, we can say that $|x-y| + |y| \geq |x|$ which we can reorder to be $|x-y| \geq |x| - |y|$.
        So we have shown the desired inequality for all real numbers $x$ and $y$.
    }
\end{exercise}

% hw problem 3 -----------------------------------------------------------------

\begin{exercise}{2.3.3}{1}
    \problem{
        Write out a proof that $lim_{k \to \infty} (x_k + y_k) = x + y$ if $ lim_{k \to \infty} x_k = x$ and $lim_{k \to \infty} y_k = y$ for sequences of real numbers.
    }
    \proof{
        We want to show (for sequences of real numbers) that the limit of a sum is the same as the sum of the limits.
        Given that $ lim_{k \to \infty} x_k = x$ and $lim_{k \to \infty} y_k = y $, we want to show that $\{ x_k + y_k \}$ satisfies the Cauchy criterion and converges to $x + y$. \parspace
        Saying that $\{ x_k \}$ and $\{ y_k \}$ have limits is equivalent to saying $\{ x_k \}$ and $\{ y_k \}$ are Cauchy.
        We also know the sum of two Cauchy sequences to be Cauchy, so certainly $\{ x_k + y_k \}$ is Cauchy. \parspace
        But what is the limit of $\{ x_k + y_k \}$?
        Let's begin by noting that $lim_{k \to \infty} x_k = x \implies $ $\forall n, \exists m_a$ such that $|x_a - x| \leq 1/2n$ for all $a \geq m_a$.
        Similarly $lim_{k \to \infty} y_k = y \implies $ $\forall n, \exists m_b$ such that $|y_b - y| \leq 1/2n$ for all $b \geq m_b$.
        Choose $k = max\{ a, b \} $.
        Then $| (x_k + y_k) - (x + y)| = |(x_k - x) + (y_k - y)| \leq 1/2n + 1/2n = 1/n$ which satisfies the definition of a limit.
        Thus the limit of a sum is equal the sum of the limits.
    }
\end{exercise}

% hw problem 4 -----------------------------------------------------------------

\begin{exercise}{2.3.3}{3}
    \problem{
        Let $x_1, x_2, ...$ be a sequence of real numbers such that $|x_n| \leq 1/2^n$, and set $y_n = x_1 + x_2 + \triplecdot + x_n$. Show that the sequence $y_1, y_2, ...$ converges.
    }
    \proof{
        We must show that $y_1, y_2, ...$ converges.
        To this end, we introduce the following result for any natural number $n$: $1/2^n = \sum_{k=n+1}^\infty 1/2^k$.
        Let's now prove this result.
        First note that for some natural number $a$, $1/2^{a+1} = 1/2 * 1/2^a$.
        Then for any $a$, we have
        $$1/2^a + 1/2^{a+1} + 1/2^{a+2} + \triplecdot = 1/2^a + 1/2*1/2^a + 1/2 * 1/2^{a+1} + \triplecdot = 1/2^a + 1/2 * (1/2^a + 1/2^{a+1} + \triplecdot) $$
        which implies that
        $$1/2^a = 1/2 * (1/2^a + 1/2^{a+1} + \triplecdot) = 1/2 * 1/2^a + 1/2 * 1/2^{a+1} + \triplecdot = 1/2^{a+1} + 1/2^{a+2} + \triplecdot = \sum_{k=a+1}^\infty 1/2^k $$
        Now to show that $y_1, y_2, ...$ converges, we will show that $y_1, y_2, ...$ is Cauchy.
        So, given a natural number $n$, we must show the existence of an index $m$ such that $|y_j - y_k| \leq 1/n$ for all $j,k \geq m$.
        For any index $m$, we can provide an upper bound for $|y_j - y_k|$ by choosing the largest possible $y_j$ and the smallest possible $y_k$.
        Because $1/2^n = \sum_{k=n+1}^\infty 1/2^k$ and $x_n \leq 1/2^n$, we can say that the largest $y_j$ could be is $y_m + 1/2^{m}$ and the smallest $y_k$ could be is $y_m - 1/2^m$.
        Then
        $$|y_j - y_k| \leq |(y_m + 1/2^{m}) - (y_k - 1/2^{m})| = |2/2^m| = 1/2^{m-1} $$
        Choosing $m = n$, we can say that $1/2^{m-1} = 1/2^{n-1} \leq 1/n$ for all $n$.
        So there exists an index $m$ such that $|y_j - y_k| \leq 1/n$ for all $j,k \geq m$.
        Thus $y_1, y_2, ...$ satisfies the Cauchy criterion and must also converge.
    }
\end{exercise}

\end{document}
