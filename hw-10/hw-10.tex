\documentclass[11pt]{article}

\usepackage{../analysis}

\begin{document}

\coverpage{10}

% hw problem 1 -----------------------------------------------------------------

\begin{exercise}{5.3.4}{1}
    \problem{
        Define
        $$ x_+ = \begin{cases} x & \text{if } x \geq 0 \\ 0 & \text{if } x < 0 \end{cases} $$
        Prove that $f(x) = x_+^k$ is continuously differentiable if $k$ is an integer greater than one.
    }
    \proof{
        We must show that $f(x) = x_+^k$ for $k > 1$ is continuously differentiable.
        That is, we must show $f'(x)$ exists and is continuous on $\R$.
        First observe that $f(x)$ is certainly continuously differentiable for $x \neq 0$.
        This is because for every $x \neq 0$, there exists a neighborhood of $x$ such that $f(x) = x^k$ or $f(x) = 0$ which are both continuously differentiable. \parspace
        All that remains is to show that $f(x)$ is continuously differentiable at $x = 0$.
        First, $f(x)$ is differentiable at 0 if there exists $f'(0) \in \R$ such that for all $1/m$ there exists a $1/n$ such that we have $\left| (f(x) - f(0)) / (x - 0) - f'(0) \right| \leq 1/m$ for $x \in (-1/n, 1/n)$ and $x \neq 0$.
        To find $f'(0)$, we simplify:
        $$ \left| \dfrac{f(x) - f(0)}{x - 0} - f'(0) \right|
        = \left| \dfrac{f(x)}{x} - f'(0) \right| \leq 1/m $$
        Then we split into two cases: we have $x < 0$ and $ x > 0$ (we assumed $x \neq 0$).
        If $x < 0$, then $f(x)/x = 0/x = 0$ and we must have $|f'(0)| \leq 1/m$.
        The only real number to satisfy this is $f'(0) = 0$.
        Will $f'(0) = 0$ also satisfy the inequality for $x > 0$?
        Select $n = m^{1/k-1}$.
        Then we have $| f(x)/x - f'(0) | = | x^k / x - 0 | = | x^{k-1}| \leq 1/n^{k-1} = 1/m $.
        Then $f'(0) = 0$ satisfies the derivative at 0. \parspace
        We now must verify that $f'(x)$ is continuous at 0.
        Note that
        $$ f'(x) = \begin{cases} kx^{k-1} & \text{if } x > 0 \\ 0 & \text{if } x \leq 0 \end{cases} $$
        We must show that for all $1/m$, there exists a $1/n$ such that $| f'(x) - f'(0) | = |f'(x)| \leq 1/m$ for $x \in (-1/n, 1/n)$.
        Again, we split into two cases: $x \leq 0$ and $ x > 0$.
        In the case where $x \leq 0$, we have $f'(x) = 0$ so the inequality is trivially satisfied.
        Then in the case where $x > 0$, take $n = (km)^{1/(k-1)}$.
        Then $| f'(x) | < k/n^{k-1} = 1/m$ which shows that $f'(x)$ is continuous.
        Since $f'(x)$ exists and is continuous, the function $f(x)$ for $k > 1$ is continuously differentiable.
    }
\end{exercise}

% hw problem 2 -----------------------------------------------------------------

\begin{exercise}{5.2.4}{1}
    \problem{
        Suppose $f'(x_0) = 0, f''(x_0) = 0, ... f^{n-1} (x_0) = 0$ and $f^{(n)}(x_0) > 0$ for a $C^n$ function $f$.
        Prove that $f$ has a local minimum at $x_0$ if $n$ is even and that $x_0$ is neither a local maximum nor a local minimum if $n$ is odd.
    }
    \proof{
        As in the textbook, let
        $$T_n(x) = f(x_0) + f'(x_0)(x - x_0) + \dfrac{1}{2!} f''(x_0)(x - x_0)^2 + \triplecdot + \dfrac{1}{n!} f^{(n)}(x_0)(x - x_0)^n $$
        But since $f'(x_0) = f''(x_0) = \triplecdot = f^{(n-1)}(x_0) = 0$, we have $T_n (x) = f(x_0) + \dfrac{1}{n!} f^{(n)}(x_0)(x - x_0)^n$.
        Then, by Taylor's Theorem, we must have
        $$f - T_n = o(|x - x_0|^n) \iff \lim _{x \to x_0} \dfrac{f - T_n}{|x - x_0|^n}
        = 0 $$
        which is equivalent to
        $$ \lim _{x \to x_0} \dfrac{f(x) - f(x_0) - 1/n! * f^{(n)}(x_0)(x - x_0)^n}{|x - x_0|^n} = 0 $$
        and to
        $$ \lim _{x \to x_0} \dfrac{f(x) - f(x_0)}{|x - x_0|^n} = \lim _{x \to x_0} \dfrac{1/n! * f^{(n)}(x_0)(x - x_0)^n}{|x - x_0|^n} $$
        The LHS looks like the derivative at $x_0$, note that it is slightly different because of the absolute value in the denominator.
        We now consider the two cases where $n$ is even and $n$ is odd.
        If $n$ is even then $|x - x_0|^n = (x - x_0)^n$ and we have
        $$ \lim _{x \to x_0} \dfrac{f(x) - f(x_0)}{|x - x_0|^n} = 1/n! * f^{(n)}(x_0) > 0 $$
        But then since $| x - x_0 |^n$ is strictly positive, we must have $f(x) > f(x_0)$ to satisfy the equality as $x \to x_0$ and $x \neq x_0$.
        Then there must exist a neighborhood of $x_0$ such that $f(x) > f(x_0)$ for all $x \neq x_0$ in the neighborhood.
        But then $f$ has a strict local minimum at $x_0$ as desired. \parspace
        Now consider the case where $n$ is odd.
        Then $|x - x_0|^n = (x - x_0)^n$ for $x > x_0$ and $|x - x_0|^n = - (x - x_0)^n$ fro $x < x_0$.
        So for $x < x_0$, we must have
        $$ \lim _{x \to x_0^-} \dfrac{f(x) - f(x_0)}{|x - x_0|^n} = - 1/n! * f^{(n)}(x_0) < 0 $$
        which means $f(x) < f(x_0)$ for $x < x_0$ in a neighborhood of $x_0$.
        And for $x > x_0$, we can use the logic from the even case of $n$ to show that $f(x) > f(x_0)$ in a neighborhood of $x_0$. \parspace
        So we have just shown that, for some neighborhood of $x_0$, $f(x)$ is strictly less than $f(x_0)$ when $x < x_0$ and $f(x)$ is strictly greater than $f(x_0)$ when $x > x_0$.
        Therefore, $x_0$ cannot be either of a local minimum or a local maximum.
    }
\end{exercise}

\end{document}
