\documentclass[11pt]{article}

\usepackage{../analysis}

\begin{document}

\coverpage{2}

% hw problem 1 -----------------------------------------------------------------

\begin{exercise}{1}{}
    \problem{
        Prove that between any two distinct raional numbers there are infinitely many other rationals.
    }
    \proof{
        We must show that between any two distinct rationals there exist infinitely many other rationals.
        We will prove this by contradiction, assume that there exist two distinct rational numbers $a,b$ such that there are finitely many rationals between $a$ and $b$.
        Without loss of generality, assume $ a < b $. \parspace
        We define the set $ \Q_{ab} := \{ q \in \Q \mid a < q \leq b \} $.
        Since there are finitely many rationals between $a$ and $b$, $\Q_{ab}$ has a finite number of elements.
        Any finite set of rationals has a minimum, so pick $p'/q'$ to be an expression of the rational number $ min \{ \Q_{ab} \} $.
        Since $p'/q'$ is the smallest element of $\Q_{ab}$, there are no rationals between $a$ and $p'/q'$.
        We can also conclude that for $p/q$ (some expression of $a$), $ p/q < p'/q' \iff pq' < p'q $. \parspace
        From here, we construct
        $$ x = \dfrac{pq' + p'q}{2qq'}$$
        Certainly $x$ must be rational by closure of integers under addition and multiplication.
        Since $x$ is rational, it cannot be between $p/q$ and $p'/q'$ so $x$ must satisfy one of the following cases. \parspace
        Case $x \leq p/q$:
        $$ \dfrac{pq' + p'q}{2qq'} \leq p/q \iff pq' + p'q \leq 2pq' \iff p'q \leq pq' \iff pq' \geq p'q $$
        which is a contradiction since $pq' < p'q$. \parspace
        Case $x \geq p'/q'$:
        $$ \dfrac{pq' + p'q}{2qq'} \geq p'/q' \iff pq' + p'q \geq 2p'q \iff pq' \geq p'q $$
        which is a contradiction since $pq' < p'q$. \parspace
        Since both cases for $x$ lead to a contradiction, we must have incorrectly assumed that there exist distinct rationals with finitely many rationals between them.
        Therefore, it must be true that between any two distinct rationals there are infinitely many other rationals.
    }
\end{exercise}

% hw problem 2 -----------------------------------------------------------------

\begin{exercise}{2.1.3}{3}
    \problem{
        What kinds of real numbers are representable by Cauchy sequences of integers?
    }
    \proof{
        We must say what types of real numbers are representable by Cauchy sequences of integers.
        Let $x_n = x_1, x_2, ... $ be a Cauchy sequence of integers.
        Since $x_n$ is Cauchy, it must be true that $ \forall n \in \N, \exists m \in \N \text{ such that } \forall j,k \geq m \hspace{0.5em} |x_j - x_k| \leq 1/n $. \parspace
        Choose $n = 2$, then $1/n=1/2$.
        We must also have $x_j$ (an integer) and $| x_j - x_k | \leq 1/2$, but the only integer value of $x_k$ that satisifes $| x_j - x_k | \leq 1/2$ is $x_k = x_j$.
        So, the sequence must be have a constant integer value $x_j$ beyond the $m^{th}$ term.
        Then $ x_n \rightarrow x_j $.
        Since $x_j$ is an integer, a Cauchy sequence of integers can only represent an integer.
    }
\end{exercise}

% hw problem 3 -----------------------------------------------------------------

\begin{exercise}{2.1.3}{5}
    \problem{
        Prove that if a Cauchy sequence $ x_1, x_2, ... $ of rationals is modified by changing a finite number of terms, the result is an equivalent Cauchy sequence.
    }
    \proof{
        We must show that modifying a finite number of terms in a Cauchy sequence of rationals results in an equivalent Cauchy sequence.
        Suppose we have a Cauchy sequence of rationals $ x_n = x_1, x_2, ...$ and we modify a finite number of terms to produce another sequence $x'_n = x'_1, x'_2, ...$
        Is $x'_n$ Cauchy? \parspace
        Since we modified a finite number of terms, let $x'_l$ be the last term in $x'_n$ such that $x_l \neq x'_l$.
        Let $m(n)$ be the index of the term in $x_n$ that satisfies the Cauchy criterion for error $1/n$.
        For $x'_n$, we say that $m'(n) = max \{ m(n), l+1 \}$ to provide an index that satisfies the Cauchy criterion for error $1/n$.
        So we know that $x'_n$ is Cauchy. \parspace
        But are $x_n$ and $x'_n$ equivalent?
        The sequences $x_n$ and $x'_n$ are equivalent if $\forall n \in \N, \exists m \in \N \text{ such that } \forall j \geq m \hspace{0.5em} |x_j - x'_j| \leq 1/n $.
        We have already identified that $x'_l$ is the last term in $x'_n$ that differs from $x_n$.
        So, regardless of what $n$ is chosen, we select $m = l + 1$.
        Then, $x_j = x'_j \implies |x_j - x'_j| = |x_j - x'_j| = |0| = 0 $ which is certainly less than $1/n$.
        So the sequences $x_n$ and $x'_n$ are equivalent. \parspace
        Therefore, modifying a finite number of terms in a Cauchy sequence of rationals results in an equivalent Cauchy sequence.
    }
\end{exercise}

% hw problem 4 -----------------------------------------------------------------

\begin{exercise}{2.1.3}{8}
    \problem{
        Can a Cauchy sequence of positive rational numbers be equivalent to a Cauchy sequence of negative rational numbers?
    }
    \proof{
        We will show that a Cauchy sequence of positive rational numbers can be equivalent to a Cauchy sequence of negative numbers by providing an example of two such sequences.
        We define the two sequences $ x_n^+ $ and $ x_n^- $ where the $l^{th}$ terms of $x_n^+$ and $x_n^-$ are $1/l$ and $-1/l$ respectively. \parspace
        Before showing the equivalence of $x_n^+$ and $x_n^-$, we must show that both sequences are Cauchy.
        A sequence $x$ is Cauchy if $\forall n \in \N, \exists m \in \N \text{ such that } \forall j,k \geq m \hspace{0.5em} |x_j - x_k| \leq 1/n$.
        For both $x_n^+$ and $x_n^-$, select $m = n$ to satisfy the Cauchy criterion. \parspace
        The sequences $x_n^+$ and $x_n^-$ are equivalent if $\forall n \in \N, \exists m \in \N \text{ such that } \forall k \geq m \hspace{0.5em} |x_k^+ - x_k^-| \leq 1/n$.
        Given an $n$, take $m = 2n$.
        Then $\forall k \geq m$,
        $$ |x_k^+ - x_k^-| = |x_k^+ - -x_k^+| = |x_k^+ + x_k^+| = 2x_k^+ \leq 2/(2n) = 1/n $$
        Since there exists an $m$ such that the difference between terms in the sequence is bounded by $1/n$, the sequences $x_n^+$ and $x_n^-$ are equivalent.
        Since the two sequences are equivalent, we have shown the existence of a Cauchy sequence of positive rational numbers that is equivalent to a Cauchy sequence of negative rational numbers.
    }
\end{exercise}

% hw problem 5 -----------------------------------------------------------------

\begin{exercise}{2.1.3}{9}
    \problem{
        Show that if $ x_1, x_2, ... $ is a Cauchy sequence of rational numbers there exists a positive integer $N$ such that $ x_j \leq N $ for all $j$.
    }
    \proof{
        We must show that for every Cauchy sequence of rational numbers, there exists a natural number larger than every term in the sequence.
        Take the Cauchy sequence $x_1, x_2, ...$
        Then, for every $n \in \N$ there must exist $m \in \N$ such that $\forall j,k \geq m \hspace{0.5em} |x_j - x_k| \leq 1/n$. \parspace
        Pick $n = 1$ and let $m$ be an index that satisfies the Cauchy criterion for $x_n$.
        Then define the finite set $L := \{ x_k \mid k < m \}$.
        That is, the set of sequence elements prior to $x_m$.
        Sequence elements following (and including) $x_m$ must be smaller than $\lceil x_m + 1 \rceil$.
        Then we can take $N = max \{ L, \lceil x_m + 1 \rceil \}$, which must be larger than every term in the sequence. \parspace
        So we have shown the existence of a natural number larger than every term in a Cauchy sequence.
    }
\end{exercise}

\end{document}
