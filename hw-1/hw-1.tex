\documentclass[11pt]{article}

\usepackage{../analysis}

\begin{document}

\coverpage{1}

% hw problem 1 -----------------------------------------------------------------

\begin{exercise}{1.1.3}{2c}
    \problem{
        Begin with the statement ``Multiplication of integers is associative.''
        Rewrite the state with explicit quantifiers.
        Then form the negation of the statement.
        Finally, recast the negation in a form similar to the original statement.
    }
    \proof{
        The statement ``Multiplication of integers is associative'' can be rewritten with explicit qunatifiers as
        $$ \forall a,b,c \in \Z, a * ( b * c ) = ( a * b ) * c $$
        The negation of the statement is
        $$ \exists a,b,c \in \Z \text{ such that } a * ( b * c ) \neq ( a * b ) * c $$
        We can rewrite the negation (in words) as ``Multiplication of integers is not always associative.''
    }
\end{exercise}

% hw problem 2 -----------------------------------------------------------------

\begin{exercise}{1.2.3}{1}
    \problem{
        Prove that every subset of $\N$ is either finite or countable.
        Conclude from this that there is no infinite set with cardinality less than that of $\N$.
    }
    \proof{

    }
\end{exercise}

% hw problem 3 -----------------------------------------------------------------

\begin{exercise}{1.2.3}{3}
    \problem{
        Prove that the rational numbers are countable.
    }
    \proof{
        We must show that the rational numbers are countable.
        We can do so by listing all the rationals.
        Before doing so, let's set up some notation.
        For a given $ k \in \N $, let $ Q_k = \{ \pm j/k \mid j \in \N \} $.
        We then say that $ U = \bigcup _{k=1} ^{\infty}  Q_k $.
        Then the set of rational numbers is $\Q = \{ 0 \} \bigcup U $. \parspace
        Note that each $Q_k$ is countable since we can list all of its elements as
        $$ \dfrac{1}{k}, \dfrac{-1}{k}, \dfrac{2}{k}, \dfrac{-2}{k}, \dfrac{3}{k}, \dfrac{-3}{k}, \dfrac{4}{k}, \dfrac{-4}{k}, \cdot \cdot \cdot $$
        We can then form an infinite table with $Q_k$ as the $k^{th}$ row (we denote the $i^{th}$ element of $Q_k$ as $q_{ki}$).
        \begin{center}
            \begin{tabular}{c c c c c}
                $q_{11}$ & $q_{12}$ & $q_{13}$ & $ \cdot \cdot \cdot $ \\
                $q_{21}$ & $q_{22}$ & $q_{23}$ & $ \cdot \cdot \cdot $ \\
                $q_{31}$ & $q_{32}$ & $q_{33}$ & $ \cdot \cdot \cdot $ \\
                $\cdot$  & $\cdot$                                     \\
                $\cdot$  &          & $\cdot$                          \\
                $\cdot$  &          &          & $\cdot$
            \end{tabular}
        \end{center}
        We can then list all the elements of $\Q$ by reading the above table diagonally and ignoring any duplicate elements (seperately, we must remember $0$ at the beginning):
        $$ L_\Q = 0, q_{11}, q_{21}, q_{12}, q_{31}, q_{22}, q_{23}, \triplecdot $$
        As a final step, all the duplicates in $L_\Q$ must be removed.
        Then $\N$ has a one-to-one correspondence with $L_\Q$.
        So, we have shown that $L_\Q$ is countable.
    }
\end{exercise}

% hw problem 4 -----------------------------------------------------------------

\begin{exercise}{1.2.3}{4}
    \problem{Show that if a countable subset is removed from an uncountable set, the remainder is still uncountable.}
    \proof{
        We begin by giving the problem some notation.
        Given an uncountable set $A$ and a countable subset $B \subset A$, we must show that $A \setminus B$ is uncountable. \parspace
        Towards a contradiction, let's assume that we have an uncountable set $A$ and a countable subset $B \subset A$ such that $A \setminus B$ is countable.
        Let $C = A \setminus B$. Since $B$ and $C$ are both countable, we can list all of their elements: $b_1, b_2, b_3, ...$ and $c_1, c_2, c_3, ...$.
        From here, we can also deduce that $B \bigcup C$ is countable because every element of $B \bigcup C$ can be listed: $b_1, c_1, b_2, c_2, b_3, c_3, ...$ \parspace
        Yet, $B \bigcup C = A$ and we know $A$ to be uncountable.
        So, we have reached a contradiction ($A$ cannot be both countable and uncountable). \parspace
        Since we reached a contradiction, we must have incorrectly assumed that $A \setminus B$ could be countable.
        Therefore, $A \setminus B$ must be uncountable and we have shown that removing a countable set from an uncoubtable set must always result in an uncountable set.
    }
\end{exercise}

% hw problem 5 -----------------------------------------------------------------

\begin{exercise}{1.2.3}{5}
    \problem{Let $ A_1, A_2, A_3, ... $ be countable sets, and let their Cartesian product $ A_1 \times A_2 \times A_3 \times \cdot \cdot \cdot $ be defined to be the set of all sequences $( a_1, a_2, ... )$ where $ a_k $ is an element of $ A_k $. Prove that the Cartesian product is uncountable. Show that the same conclusion holds if each of the sets $ A_1, A_2, ... $ has at least two elements.}
    \proof{
        We must show that the Caresian product of a countable number of countable sets is uncountable.
        Towards a contradiction, let's assume not.
        That is, let's assume that $P = A_1 \times A_2 \times A_3 \times \triplecdot$ is a countable set. \parspace
        Since $P$ is countable, there exists a bijective map $f: \N \longrightarrow P$.
        For a given $n \in \N$, we say $f(n) = ( a_1^n, a_2^n, a_3^n, ... )$.
        To get our contradiction, we will construct an element $p = (p_1, p_2, p_3, ...) \in P$ such that $p \notin im(f)$.
        We choose $p_k$ such that $p_k \in A_k$ and $p_k \neq a_k^k$.
    }
\end{exercise}

\end{document}
