\documentclass[11pt]{article}

\usepackage{analysis}

\begin{document}

\coverpage{1}

% hw problem 1 -----------------------------------------------------------------

\begin{exercise}{1.1.3}{2c}
    \problem{
        Begin with the statement ``Multiplication of integers is associative.''
        Rewrite the state with explicit quantifiers.
        Then form the negation of the statement.
        Finally, recast the negation in a form similar to the original statement.
    }
    \proof{
        The statement ``Multiplication of integers is associative'' can be rewritten with explicit qunatifiers as
        $$ \forall a,b,c \in \Z, a * ( b * c ) = ( a * b ) * c $$
        The negation of the statement is
        $$ \exists a,b,c \in \Z \text{ such that } a * ( b * c ) \neq ( a * b ) * c $$
        We can rewrite the negation (in words) as ``Multiplication of integers is not always associative.''
    }
\end{exercise}

% hw problem 2 -----------------------------------------------------------------

\begin{exercise}{1.2.3}{1}
    \problem{
        Prove that every subset of $\N$ is either finite or countable.
        Conclude from this that there is no infinite set with cardinality less than that of $\N$.
    }
    \proof{

    }
\end{exercise}

% hw problem 3 -----------------------------------------------------------------

\begin{exercise}{1.2.3}{3}
    \problem{
        Prove that the rational numbers are countable.
    }
    \proof{
        We can prove that the rational numbers are countable by providing a one-to-one correspondence from $\N$ to $\Q$.
        Before showing that $\Q$ is countable, we give some notation.
        For a given $ k \in \N $, let $ Q_k = \{ \pm j/k \mid j \in \N \} $.
        We then say that $ U = \bigcup _{k=1} ^{\infty}  Q_k $.
        Then we can write the set of rational numbers as $\Q = \{ 0 \} \bigcup U $. \parspace
        Note that each $Q_k$ is countable. % give the correspondence (f(j) = something with floor or ceil?)
    }
\end{exercise}

% hw problem 4 -----------------------------------------------------------------

\begin{exercise}{1.2.3}{4}
    \problem{Show that if a countable subset is removed from an uncountable set, the remainder is still uncountable.}
    \proof{}
\end{exercise}

% hw problem 5 -----------------------------------------------------------------

\begin{exercise}{1.2.3}{5}
    \problem{Let $ A_1, A_2, A_3, ... $ be countable sets, and let their Cartesian product $ A_1 \times A_2 \times A_3 \times \cdot \cdot \cdot $ be defined to be the set of all sequences $( a_1, a_2, ... )$ where $ a_k $ is an element of $ A_k $. Prove that the Cartesian product is uncountable. Show that the same conclusion holds if each of the sets $ A_1, A_2, ... $ has at least two elements.}
    \proof{}
\end{exercise}

\end{document}
