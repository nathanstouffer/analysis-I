\documentclass[11pt]{article}

\usepackage{../analysis}

\begin{document}

\coverpage{7}

% hw problem 1 -----------------------------------------------------------------

\begin{exercise}{4.1.5}{2}
    \problem{
        Let $A$ be the set defined by the equations $f_1 (x) = 0, f_2 (x) = 1, ... , f_n (x) = 0,$ where $f_1, ..., f_n$ are continuous functions defined on the whole line.
        Show that $A$ is closed.
        Must $A$ be compact?
    }
    \proof{
        We first show that $A$ is closed.
        Given a function $f^k$, let $D_k = \{ x \mid f^k (x) = 0 \}$.
        Then we can say that $A = \cup _{i=1}^n D_i$.
        So $A$ is union of a finite number of sets.
        If each $D_k$ is closed, then $A$ (a finite union of closed sets) must also be closed. \parspace
        Let's now verify that every $D_k$ is closed.
        Pick an arbitrary $D_k = \{ x \mid f^k (x) = 0 \}$.
        Now consider the result (from exercise 4.1.5 problem 1) that a function $f$ defined on a closed domain is continuous if and only if the inverse image of every closed set is a closed set.
        $f^k$ is a continuous function (by hypothesis) defined on the closed set $\R$, so we can equivalently say that the inverse image of every closed set is a closed set.
        Since $\{ 0 \}$ is a closed set, the inverse image of $\{ 0 \}$ under $f^k$ is closed.
        But the inverse image is $\{ x \mid f^k (x) = 0 \}$ which is $D_k$.
        Therefore $D_k$ is closed which implies that $A$ is closed. \parspace
        It is not necessarily the case that $A$ is compact.
        We already know $A$ is necessarily closed so let's find a counterexample where $A$ is not bounded (since then $A$ will not be compact).
        Suppose we have $f^1$ be the constant function $f^1: x \mapsto 0$.
        Then $D_1 = \{ x \mid f^1 (x) = 0 \} = \R$ which is not bounded.
        Therefore $A$ is not necessarily compact.
    }
\end{exercise}

% hw problem 2 -----------------------------------------------------------------

\begin{exercise}{4.1.5}{4}
    \problem{
        Give a definition of $\lim _{x \to \infty} f(x) = y$.
        Show that this is true if and only if for every sequence $x_1, x_2, ...$ of points in the domain of $f$ such that $\lim _{n \to \infty} x_n = + \infty $, we have $\lim _{n \to \infty} f(x_n) = y$.
    }
    \proof{
        We define $\lim _{x \to \infty} f(x)$ (for functions with unbounded positive domain) to be $y \in \R$ such that for all $m$ there exists $n$ such that for all $x > n$ ($x \in \R$) we have $| y - f(x) | < 1/m$.
        We now show that this is the case if and only if every sequence of points in the domain with a limit of $+ \infty$ has $\lim _{n \to \infty} f(x_n) = y$. \parspace
        First suppose that $y$ is a real number such that for every $m$ there exists $n$ such that for all $x > n$ we have $| y - f(x) | < 1/m$.
        Then we would like to show every sequence $x_1, x_2, ...$ in the domain of $f$ with $+ \infty$ as a limit has $\lim _{n \to \infty} f(x_n) = y$.
        The sequence $x_1, x_2, ... \to +\infty$ so for every $a \in \N$ there exists an index $b$ such that for all $j \geq b$ we have $x_j > a$.
        But then taking $a = n$ we have $|y - f(x_j)| < 1/m$ for all $j \geq n$.
        In the limit $j \to \infty$, we have $| y - f(x) | \leq 1/m$ which satisfies $\lim _{n \to \infty} f(x_n) = y$. \parspace
        Now suppose that every sequence of points in the domain of $f$ such that $\lim _{x \to \infty} x_n = + \infty$, we have $\lim _{n \to \infty} f(x_n) = y$.
        We want to show that $\lim _{x \to \infty} f(x) = y$.
        By Theorem 4.1.1, what we just said is equivalent to claiming that $\lim _{x \to \infty} f(x)$ exists.
        Further, the theorem states that every sequence $f(x_1), f(x_2), ...$ has a common limit and that common limit the limit of the function.
        Therefore $\lim _{x \to \infty} f(x) = y$.
    }
\end{exercise}

% hw problem 3 -----------------------------------------------------------------

\begin{exercise}{4.1.5}{7}
    \problem{
        Give an example of a continuous function with domain $\R$ such that the inverse image of a compact set is not compact.
    }
    \proof{
        Consider the example I gave in Problem 2 where $f$ is the constant function that takes every real number to 0.
        Certainly $f$ is continuous and the compact set 0 has $\R$ for an inverse image under $f$.
        Although $\R$ is closed, it is not bounded so $\R$ is not compact.
    }
\end{exercise}

% hw problem 4 -----------------------------------------------------------------

\begin{exercise}{4.1.5}{10}
    \problem{
        Show that a function that satisfies a Lipschitz condition is uniformly continuous.
    }
    \proof{
        Suppose we have some function $f$ such that $ | x - x_0 | < 1/Mm $ implies that $ | f(x) - f(x_0) | < 1/m$ for some constant $M$, is $f$ uniformly continuous?
        The function $f$ is uniformly continuous if for all $m$, there exists a $n$ such that $ | x - x_0 | < 1/n$ implies that $ |f(x) - f(x_0)| < 1/m $ for all $x,x_0 \in D$ satisfying $| x - x_0 | < 1/n$.
        From the fact that $f$ satisfies a Lipschitz condition, we can gather that $| f(x) - f(x_0) | \leq M | x - x_0 |$.
        Then we can just take $n = Mm$ to satisfy uniform continuity: $ |f(x) - f(x_0)| < M| x - x_0 | = M*1/Mm = 1/m$.
        Therefore $f$ must also be uniformly continuous.
    }
\end{exercise}

\end{document}
