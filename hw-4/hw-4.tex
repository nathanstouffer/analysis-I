\documentclass[11pt]{article}

\usepackage{../analysis}

\begin{document}

\coverpage{4}

% hw problem 1 -----------------------------------------------------------------

\begin{exercise}{3.1.3}{1}
    \problem{
        Compute the sup, inf, limsup, liminf, and all the limit points of the sequence $x_1, x_2, ...$ where $x_n = 1/n + (-1)^n$.
    }
    \proof{

    }
\end{exercise}

% hw problem 2 -----------------------------------------------------------------

\begin{exercise}{3.1.3}{2}
    \problem{
        If a bounded sequence is the sum of a monotone increasing and a monotone decreasing sequence ($x_n = y_n + z_n$ where $\{ y_n \}$ is monotone increasing and $\{ z_n \}$ is monotone decreasing), does it follow that the sequence converges? What if $\{ y_n \}$ and $\{ z_n \}$ are bounded?
    }
    \proof{

    }
\end{exercise}

% hw problem 3 -----------------------------------------------------------------

\begin{exercise}{3.1.3}{4}
    \problem{
        Prove $\sup (A \cup B) \geq \sup A$ and $\sup (A \cap B) \leq \sup A$.
    }
    \proof{
        We begin by proving that $\sup (A \cup B) \geq \sup A$.
        To this end, suppose that the opposite were true: that $\sup (A \cup B ) < \sup A$.
        By definition of $\sup$, we know $\sup A = x$ where $x$ is the smallest extended real number satisfying $a \leq x$ for all  $a \in A$.
        We also know $\sup (A \cup B) = z$ where $z$ is the smallest extended real number satisfying $a \leq z$ for all $a \in A$ and $b \leq z$ for all $b \in B$.
        From this, we conclude that $\sup (A \cup B) < \sup A \iff z < x$.
        We also know $a \leq z$ for all $a$, so it must be true that $a \leq z < x$ for all $a$.
        But we have just showed that $x$ is not the smallest extended real number satisfying $a \leq x$ for all $a$, a contradiction!
        So it must be true that $\sup (A \cup B) \leq \sup A$. \parspace
        Now we show that $\sup (A \cap B) \leq \sup A$.

    }
\end{exercise}

% hw problem 4 -----------------------------------------------------------------

\begin{exercise}{3.1.3}{6}
    \problem{
        Is every subsequence of a subsequence of a subsequence also a subsequence of the sequence?
    }
    \proof{
        % yes
    }
\end{exercise}

% hw problem 5 -----------------------------------------------------------------

\begin{exercise}{3.1.3}{9}
    \problem{
        Can there exist a sequence whose set of limit points is exactly $1, 1/2, 1/3, ...$?
    }
    \proof{
        % no for 0 must be a limit point of any such sequence
    }
\end{exercise}


\end{document}
