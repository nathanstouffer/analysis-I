\documentclass[11pt]{article}

\usepackage{../analysis}

\begin{document}

\coverpage{8}

% hw problem 1 -----------------------------------------------------------------

\begin{exercise}{4.2.4}{1}
    \problem{
        If $f$ is monotone increasing on an interval and has a jump discontinuity at $x_0$ in the interior of the domain, show that the jump is bounded above by $f(x_2) - f(x_1)$ for any two points $x_1, x_2$ of the domain surrounding $x_0$, $x_1 < x_0 < x_2$.
    }
    \proof{
        For some monotone increasing function $f$ defined on an interval $I$ with a jump at $x_0$ in the interior of $I$.
        Let the height of the jump be $h = \lim _{x \to x_0^+} f(x) - \lim _{x \to x_0^-} f(x)$.
        Now suppose we have two points $x_1, x_2 \in I$ such that $x_1 < x_0 < x_2$.
        We want to show that $h \leq f(x_2) - f(x_1)$. \parspace
        Since $f$ increases monotonically, we know that $f(x_1) \leq \lim _{x \to x_0^-} f(x)$ and $\lim _{x \to x_0^+} f(x) \leq f(x_2)$, which imply that $f(x_1) + \lim _{x \to x_0^+} f(x) \leq \lim _{x \to x_0^-} f(x) + f(x_2)$.
        Equvalently, $f(x_2) - f(x_1) \geq \lim _{x \to x_0^-} f(x) - \lim _{x \to x_0^+} f(x) = h$.
        Therefore, the jump is bounded above by $f(x_2) - f(x_1)$.
    }
\end{exercise}

% hw problem 2 -----------------------------------------------------------------

\begin{exercise}{4.2.4}{3}
    \problem{
        If the domain of a continuous function is an interval, show that the image is an interval.
        Give examples where the image is an open interval.
    }
    \proof{

    }
\end{exercise}

% hw problem 3 -----------------------------------------------------------------

\begin{exercise}{4.2.4}{9}
    \problem{
        If $f$ and $g$ are uniformly continuous, show that $f + g$ is uniformly continuous.
    }
    \proof{

    }
\end{exercise}

% hw problem 4 -----------------------------------------------------------------

\begin{exercise}{4.2.4}{11}
    \problem{
        If $f$ is a continuous function on a compact set, show that either $f$ has a zero or $f$ is bounded away from zero ($| f(x) | > 1/n$ for all $x$ in the domain, for some $1/n$).
    }
    \proof{

    }
\end{exercise}

\end{document}
